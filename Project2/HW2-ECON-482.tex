\documentclass[11pt, oneside]{article}   	% use "amsart" instead of "article" for AMSLaTeX format
\usepackage{geometry}                		% See geometry.pdf to learn the layout options. There are lots.
\geometry{letterpaper}                   		% ... or a4paper or a5paper or ... 
%\geometry{landscape}                		% Activate for rotated page geometry
\usepackage[parfill]{parskip}    		% Activate to begin paragraphs with an empty line rather than an indent

\usepackage[utf8]{inputenc} % Any characters can be typed directly from the keyboard, eg éçñ
\usepackage{textcomp} % provide lots of new symbols
\usepackage{graphicx}  % Add graphics capabilities
%\usepackage{epstopdf} % to include .eps graphics files with pdfLaTeX
\usepackage{flafter}  % Don't place floats before their definition
%\usepackage{topcapt}   % Define \topcation for placing captions above tables (not in gwTeX)
%\usepackage{natbib} % use author/date bibliographic citations %May conflict with some bibliography style download somewhere


\usepackage{amsmath,amssymb}  % Better maths support & more symbols
\usepackage{bm}  % Define \bm{} to use bold math fonts

\usepackage[pdftex,bookmarks,colorlinks,breaklinks]{hyperref}  % PDF hyperlinks, with coloured links
%\definecolor{dullmagenta}{rgb}{0.4,0,0.4}   % #660066
%\definecolor{darkblue}{rgb}{0,0,0.4}
\hypersetup{linkcolor=blue,citecolor=blue,filecolor=dullmagenta,urlcolor=darkblue} % coloured links
%\hypersetup{linkcolor=black,citecolor=black,filecolor=black,urlcolor=black} % black links, for printed output

\usepackage{memhfixc}  % remove conflict between the memoir class & hyperref
% \usepackage[activate]{pdfcprot}  % Turn on margin kerning (not in gwTeX)
% \usepackage{pdfsync}  % enable tex source and pdf output syncronicity

\usepackage{array}%for table width
\newcolumntype{L}[1]{>{\raggedright\let\newline\\\arraybackslash\hspace{0pt}}m{#1}}
\newcolumntype{C}[1]{>{\centering\let\newline\\\arraybackslash\hspace{0pt}}m{#1}}
\newcolumntype{R}[1]{>{\raggedleft\let\newline\\\arraybackslash\hspace{0pt}}m{#1}}

\usepackage{caption}%Add captions to figures
\usepackage[toc,page]{appendix}%Add appendix
\usepackage{csvsimple}%Import csv file as a table
\usepackage{float}%Force figures place (H)

\usepackage{booktabs}%For \toprule \midrule \bottomrule

\usepackage{multicol}%Create multicolumns in report
\usepackage{wrapfig}%Insert floating elements into multicolumns in report
\usepackage{float}%To use placement specifier H in float element of figure
\usepackage{longtable}%required by longtabu
\usepackage{tabu}%Build a table span several pages
\usepackage{tabularx}%Control and adjust the width of columns in tabulars
\usepackage{listings}%Include code into latex
\usepackage{color}

\usepackage{pgfplotstable}% Include excel file into latex

\usepackage{tikz}%Package for drawing graphical models
\usetikzlibrary{shapes, arrows, calc, positioning,matrix}
\tikzset{
data/.style={circle, draw, text centered, minimum height=3em ,minimum width = .5em, inner sep = 2pt},
empty/.style={circle, text centered, minimum height=3em ,minimum width = .5em, inner sep = 2pt},
}
 
\definecolor{codegreen}{rgb}{0,0.6,0}
\definecolor{codegray}{rgb}{0.5,0.5,0.5}
\definecolor{codepurple}{rgb}{0.58,0,0.82}
\definecolor{backcolour}{rgb}{0.95,0.95,0.92}

\lstdefinestyle{mystyle}{
    backgroundcolor=\color{backcolour},   
    commentstyle=\color{codegreen},
    keywordstyle=\color{magenta},
    numberstyle=\tiny\color{codegray},
    stringstyle=\color{codepurple},
    basicstyle=\footnotesize,
    breakatwhitespace=false,         
    breaklines=true,                 
    captionpos=b,                    
    keepspaces=true,                 
    numbers=left,                    
    numbersep=5pt,                  
    showspaces=false,                
    showstringspaces=false,
    showtabs=false,                  
    tabsize=2
}
 
\lstset{style=mystyle}

\DeclareCaptionFont{white}{\color{white}}
\DeclareCaptionFormat{listing}
     {\parbox{\dimexpr\textwidth-4\fboxsep}{\centering #1#2#3}}
\captionsetup[lstlisting]{format=listing}
%\captionsetup[lstlisting]{format=listing,labelfont=white,textfont=white}

\usepackage{enumerate}

\usepackage{fancyvrb}%include a txt file into latex pdf

\usepackage{seqsplit}
\newcommand\foo[2]{%
    \begin{minipage}{#1}
    \seqsplit{#2}
    \end{minipage}
    }

%newcommand
\newcommand{\notimplies}{%
  \mathrel{{\ooalign{\hidewidth$\not\phantom{=}$\hidewidth\cr$\implies$}}}}

%\usepackage[utf8]{inputenc}
%\usepackage[english]{babel}
%\usepackage{amsmath}
%\usepackage{amsfonts}
%\usepackage{amssymb}
%\usepackage{amsbsy}
%\usepackage{bm}
%\usepackage{fixmath}
%\usepackage{makeidx}
%\usepackage{graphicx}
%
%\usepackage{array}%for table width
%\newcolumntype{L}[1]{>{\raggedright\let\newline\\\arraybackslash\hspace{0pt}}m{#1}}
%\newcolumntype{C}[1]{>{\centering\let\newline\\\arraybackslash\hspace{0pt}}m{#1}}
%\newcolumntype{R}[1]{>{\raggedleft\let\newline\\\arraybackslash\hspace{0pt}}m{#1}}
%
%\usepackage{grffile}%avoid showing file name when including a file
%\usepackage{verbatim}%include code as appendix
%\usepackage[left=1in,right=1in,top=1in,bottom=0.8in]{geometry}
%
%\usepackage{array}
%
%\usepackage{pdfpages}%for including pdf of R-code
%
%\setlength\parindent{0pt}%noindent
%
%
%
%\DeclareMathOperator{\Tr}{Tr}

\DeclareMathOperator*{\argmin}{argmin}


\title{HW2-ECON-482}
\author{Kungang Zhang}
\date{November 14, 2016}							% Activate to display a given date or no date

\begin{document}
\maketitle
\section{Prob 1:}
\begin{enumerate}[(1)]
\item
The SVAR model is:
\begin{align}
A_0 y_t =& A_0 + A_1 y_{t-1}+\cdots+A_{p} y_{t-p} + \varepsilon \\
y_t =& B_0 + B_1 y_{t-1}+\cdots+B_{p} y_{t-p} + A_0^{-1}\varepsilon
\label{model:SVAR}
\end{align}
Use the posterior form to the the MAP for $ \beta$ and $ A_0$. The MAP for $ \beta$ is just:
\begin{align}
\hat{ B} =& (X^TX + \Omega ^{-1}) ^{-1} (X^T Y + \Omega ^{-1} \hat{b}) \nonumber \\
=& ( \lambda^2 X^TX + \Phi ^{-1}) ^{-1} ( \lambda^2 X^T Y + \Phi ^{-1} \hat{b}) 
\label{eqn:post_B}
\end{align}
where I used the notation, $ \Omega = \lambda^2 \Phi$, and $\hat{ \beta} = vec(\hat{B})$. The definition of $ \Phi$ can be found on the last homework.
To get posterior mode of $A_0$, I need to maximize:
\begin{align}
p(A_0 | Y) \propto |A_0| ^{T+M} exp\left\{-\frac{1}{2} tr \left( \left( \hat{S} + \left( \hat{B} - \hat{b} \right)^T \Omega ^{-1} \left( \hat{B} - \hat{b} \right) \right)A_0^T A_0  \right) \right\}
\label{eqn:post_A0}
\end{align}
To maximize it, first take a log and then use global maximization toolbox in Matlab. Denote $\hat{\hat{S}} =  \hat{S} + \left( \hat{B} - \hat{b} \right)^T \Omega ^{-1} \left( \hat{B} - \hat{b}\right)$. Se the initial guess for $a_0$ as the upper Cholesky decomposition of $\hat{\hat{S}}^T\hat{\hat{S}}$ with only free variables being chosen so that they can be nonzero. Use multi-starting points to achieve somewhat ``global" optimal. By experiments, I found that the impulse response functions w.r.t. the monetary policy shock can be different for $200$ and $1000$ multi-starts. The optimal $A_0$ is as shown below (use $2500$ multi-starts).

\pgfplotstabletypeset[
    col sep=comma,
    string type,
%    columns/Name/.style={column name=Name, column type={|l}},
%    columns/Observed data/.style={column name=Observed data, column type={|c}},
%    columns/Flat prior/.style={column name=Flat prior, column type={|c}},
%    columns/Minnesota prior/.style={column name=Minnesota prior, column type={|c}},
%    columns/SOC (1)/.style={column name=SOC (1), column type={|c}},
%    columns/SOC (5)/.style={column name=SOC (5), column type={|c|}},
    columns/0/.style={column type={|l}},
    columns/1/.style={column type={l}},
    columns/2/.style={column type={l}},
    columns/3/.style={column type={l}},
    columns/4/.style={column type={l}},
    columns/5/.style={column type={l|}},
    header=false,
    every head row/.style={before row=\hline,after row=\hline},
    every last row/.style={after row=\hline},
    ]{Prob 1-A0_opt_mine.csv}
    
The mode of coefficient $B$ is as shown below.
\pgfplotstabletypeset[
    col sep=comma,
    string type,
%    columns/Name/.style={column name=Name, column type={|l}},
%    columns/Observed data/.style={column name=Observed data, column type={|c}},
%    columns/Flat prior/.style={column name=Flat prior, column type={|c}},
%    columns/Minnesota prior/.style={column name=Minnesota prior, column type={|c}},
%    columns/SOC (1)/.style={column name=SOC (1), column type={|c}},
%    columns/SOC (5)/.style={column name=SOC (5), column type={|c|}},
    columns/0/.style={column type={|l}},
    columns/1/.style={column type={l}},
    columns/2/.style={column type={l}},
    columns/3/.style={column type={l}},
    columns/4/.style={column type={l}},
    columns/5/.style={column type={l|}},
    header=false,
    every head row/.style={before row=\hline,after row=\hline},
    every last row/.style={after row=\hline},
    begin table=\begin{longtable},
    end table=\end{longtable},
    ]{Prob 1-B_hat_mine.csv}

%\begin{tabular}{| l | l | l | l | l | l |}
%\csvreader{Prob 1-A0_opt_mine.csv)}{}
%%{\\\hline\csvcoli&\csvcolii}
%\end{tabular}
%
%\begin{tabular}{| l | l | l | l | l | l |}
%\csvreader{Prob 1-B_hat_mine.csv)}{}
%%{\\\hline\csvcoli&\csvcolii}
%\end{tabular}

\item
Using companion form to construct impulse response functions.
\begin{align}
Z _{t} = \Psi + \Phi Z _{t-1} + G\xi
\end{align}
where 
\begin{align}
Z _{t} =& [y _{t}, y _{t-1}, \cdots, y _{t-p+1}] \\
\Psi =& [B_0; 0 _{(p-1)M \times 1}] \\
\Phi =& 
 \begin{bmatrix}
 B_1 & B_2 & B_3 &  \cdots & B_{p-1} & B_p \\
 I_M & 0 & 0 &  \cdots & 0 & 0 \\
 0 & I_M & 0 &  \cdots & 0 & 0 \\
 0 & 0 & \ddots &  \cdots & 0 & 0 \\
 0 & 0 & 0 & \cdots & I_M & 0
\end{bmatrix} \\
G =&  \begin{bmatrix}
 A_0^{-1} & 0 _{M \times (p-1)M} \\
 0 _{(p-1)M \times M} & 0 _{(p-1)M}
\end{bmatrix} \\
\xi =& [ \varepsilon; 0 _{(p-1)M \times 1}]
\end{align}

\begin{figure}[!ht]
\begin{center}
\includegraphics[width = 0.8\textwidth]{{"Prob 1-Error_Bands"}.jpg}
\captionsetup{width=0.8\textwidth}
\caption{The impulse response functions under monetary policy shock for projection horizon $48$. The comparison with figures in \cite{sims2006were} are very close.}
\label{fig:IMF_errbd}
\end{center}
\end{figure}

\item
Here, we use Metropolis algorithm to sample the marginal posterior of $A_0$, and then for each sample we sample a coefficient $B$. In this way, we sample from full joint posterior of $A_0$ and $B$. The $90\%$ and $68\%$ error bands are given in Figure \ref{fig:IMF_errbd}. Figure \ref{fig:trace_plots} are trace plots of this MCMC ($C = 0.5$ and $NumSim = 1,000,000$; the accepting rate is around $0.2$).

\begin{figure}[!ht]
\begin{center}
\includegraphics[width = 0.8\textwidth]{{"Prob 1-Trace_plots"}.jpg}
\captionsetup{width=0.8\textwidth}
\caption{Trace plots for free variables in $A_0$.}
\label{fig:trace_plots}
\end{center}
\end{figure}

\item
The monetary policy shock has been well identified because the these impulse response functions have tendency eventually going to zero. During simulation, we set a sign constraint to ensure that a positive monetary policy shock will result in immediate positive response of the federal funds rate.

\item
The variance decomposition is like following:
\begin{align}
VD _{i,j}(k) = \frac{\left[ \sum_{s = 0}^{s = k-1} \Phi^s G \tilde{D} G^T \Phi^{sT} \right] _{i,i}}{\left[ \sum_{s = 0}^{s = k-1} \Phi^s G D G^T \Phi^{sT} \right] _{i,i}}
\end{align}
where $\tilde{D}$ has the only nonzero entry $(j,j)$ equal to $1$ and $D$ is $M$ dimensional identity matrix. The portion of variance due to monetary policy shock is $0.133948456672977$.

\end{enumerate}

\section{Prob 2:}
\begin{enumerate}[(1)]
\item
Replace \textit{the real GDP} with \textit{the employment-population ratio}. The procedure of calculating mode and error bands is repeated (Figure \ref{fig:err_bd_2-1}). 
\begin{figure}[!ht]
\begin{center}
\includegraphics[width = 0.8\textwidth]{{"Prob 2-1-Error_Bands"}.jpg}\\
\includegraphics[width = 0.8\textwidth]{{"Prob 2-1-Trace_plots"}.jpg}
\captionsetup{width=0.8\textwidth}
\caption{IMF with error bands with \textit{the real GDP} replaced by \textit{the employment-population ratio}.}
\label{fig:err_bd_2-1}
\end{center}
\end{figure}
Results are approximately robust to these changes.

\item
Replace \textit{the real GDP} with \textit{the industrial production}. The procedure of calculating mode and error bands is repeated (Figure \ref{fig:err_bd_2-2}). 
\begin{figure}[!ht]
\begin{center}
\includegraphics[width = 0.8\textwidth]{{"Prob 2-2-Error_Bands"}.jpg}
\includegraphics[width = 0.8\textwidth]{{"Prob 2-2-Trace_plots"}.jpg}
\captionsetup{width=0.8\textwidth}
\caption{IMF with error bands with \textit{the real GDP} replaced by \textit{the employment-population ratio}.}
\label{fig:err_bd_2-2}
\end{center}
\end{figure}
Results are not approximately robust to these changes.

\item
Replace \textit{the real GDP} with \textit{the industrial production}, and \textit{the M2 divisia monetary index} with \textit{the M2 money stock}. The procedure of calculating mode and error bands is repeated (Figure \ref{fig:err_bd_2-3}). 
\begin{figure}[!ht]
\begin{center}
\includegraphics[width = 0.8\textwidth]{{"Prob 2-3-Error_Bands"}.jpg}
\includegraphics[width = 0.8\textwidth]{{"Prob 2-3-Trace_plots"}.jpg}
\captionsetup{width=0.8\textwidth}
\caption{IMF with error bands with \textit{the real GDP} replaced by \textit{the employment-population ratio}, and \textit{the M2 divisia monetary index} replaced by \textit{the M2 money stock}.}
\label{fig:err_bd_2-3}
\end{center}
\end{figure}
Results are approximately robust to these changes.

\end{enumerate}

\section{Prob 3:}
\begin{enumerate}[(1)]
\item
Use extended periods of data from $1959:1$ to $2008:12$ excluding ZLB period. Repeat the previous procedure (Figure \ref{fig:err_bd_3-1}).
\begin{figure}[!ht]
\begin{center}
\includegraphics[width = 0.8\textwidth]{{"Prob 3-1-Error_Bands"}.jpg}
\includegraphics[width = 0.8\textwidth]{{"Prob 3-1-Trace_plots"}.jpg}
\captionsetup{width=0.8\textwidth}
\caption{Use extended periods of data from $1959:1$ to $2008:12$ excluding ZLB period.}
\label{fig:err_bd_3-1}
\end{center}
\end{figure}
Results are not approximately robust to these changes.

\item
Use the entire extended periods of data from $1959:1$ to $2014:12$. Repeat the previous procedure (Figure \ref{fig:err_bd_3-2}).
\begin{figure}[H]
\begin{center}
\includegraphics[width = 0.8\textwidth]{{"Prob 3-2-Error_Bands"}.jpg}
\includegraphics[width = 0.8\textwidth]{{"Prob 3-2-Trace_plots"}.jpg}
\captionsetup{width=0.8\textwidth}
\caption{Use the entire extended periods of data from $1959:1$ to $2014:12$.}
\label{fig:err_bd_3-2}
\end{center}
\end{figure}
Results are not approximately robust to these changes.
\end{enumerate}




\bibliographystyle{report.bst}
\bibliography{ref.bib}

\begin{appendices}
\begin{lstlisting}[language=Matlab ,caption={\textit{prob1.m} solves problem 1.}, label={code:prob1}]
clear();
%Load the data set
Dat = xlsread('SZdata.xlsx');
Dat = Dat(:,2:7);
%Set constants
p = 13;%Maximum lag
lam = 0.2;%Hyperparameter
M = 6;%Dimension of vector
T = size(Dat,1)-p;%Time for forcasting
K = M*p+1;
vc = 10^6;%First element in Minnesota prior
mu = 1;
Num = 500000;%Number of MCMC simulation
C = 0.5;%Scaling parameter for the proposal distribution variation
ir_t = 48;%Impulse response function horizon
mp_shock = zeros(M,1);%Monetary policy shock
mp_shock(M) = 1;
multistarts = 2500;%Number of starts
t_proj = 36;%Variance decomposition
text = 'Prob 1-';
IMF_Errbd(Dat,p,lam,M,T,K,vc,mu,Num,C,ir_t,mp_shock,multistarts,t_proj,text);
\end{lstlisting}

\begin{lstlisting}[language=Matlab ,caption={\textit{prob2.m}}, label={code:prob2}]
clear();
%Load the data set
Dat_all = xlsread('SZdata.xlsx');
Dat = Dat_all(:,[8,3:7]);
%Set constants
p = 13;%Maximum lag
lam = 0.2;%Hyperparameter
M = 6;%Dimension of vector
T = size(Dat,1)-p;%Time for forcasting
K = M*p+1;
vc = 10^6;%First element in Minnesota prior
mu = 1;
Num = 500000;%Number of MCMC simulation
C = 0.5;%Scaling parameter for the proposal distribution variation
ir_t = 48;%Impulse response function horizon
mp_shock = zeros(M,1);%Monetary policy shock
mp_shock(M) = 1;
multistarts = 2500;%Number of starts
t_proj = 36;%Variance decomposition
text = 'Prob 2-1-';
% Replace GDP with employment-population ratio
IMF_Errbd(Dat,p,lam,M,T,K,vc,mu,Num,C,ir_t,mp_shock,multistarts,t_proj,text);
% Replace GDP with industrial production
Dat = Dat_all(:,[9,3:7]);
text = 'Prob 2-2-';
IMF_Errbd(Dat,p,lam,M,T,K,vc,mu,Num,C,ir_t,mp_shock,multistarts,t_proj,text);
% Replace GDP with industrial produciton, and M2 divisia monetary index
% with M2 money stock
Dat = Dat_all(:,[9,3:5,10,7]);
text = 'Prob 2-3-';
IMF_Errbd(Dat,p,lam,M,T,K,vc,mu,Num,C,ir_t,mp_shock,multistarts,t_proj,text);
\end{lstlisting}

\begin{lstlisting}[language=Matlab ,caption={\textit{prob3.m}}, label={code:prob3}]
clear();
%Load the data set
Dat_all = xlsread('SZdataExtended.xlsx');
Dat = Dat_all(:,2:7);
%Set constants
p = 13;%Maximum lag
lam = 0.2;%Hyperparameter
M = 6;%Dimension of vector
T = size(Dat,1)-p;%Time for forcasting
K = M*p+1;
vc = 10^6;%First element in Minnesota prior
mu = 1;
Num = 500000;%Number of MCMC simulation
C = 0.5;%Scaling parameter for the proposal distribution variation
ir_t = 48;%Impulse response function horizon
mp_shock = zeros(M,1);%Monetary policy shock
mp_shock(M) = 1;
multistarts = 2500;%Number of starts
t_proj = 36;%Variance decomposition
text = 'Prob 3-2-';
% Use the entire data set
IMF_Errbd(Dat,p,lam,M,T,K,vc,mu,Num,C,ir_t,mp_shock,multistarts,t_proj,text);
% Excluding ZLB period
Dat = Dat_all(1:600,2:7);
text = 'Prob 3-1-';
IMF_Errbd(Dat,p,lam,M,T,K,vc,mu,Num,C,ir_t,mp_shock,multistarts,t_proj,text);
\end{lstlisting}

\begin{lstlisting}[language=Matlab ,caption={\textit{prob2.m}}, label={code:prob2}]
clear();
%Load the data set
Dat = xlsread('SZdata.xlsx');
Dat = Dat(:,[8,3:7]);
%Set constants
p = 13;%Maximum lag
lam = 0.2;%Hyperparameter
M = 6;%Dimension of vector
T = size(Dat,1)-p;%Time for forcasting
K = M*p+1;
vc = 10^6;%First element in Minnesota prior
mu = 1;
Num = 500000;%Number of MCMC simulation
C = 0.5;%Scaling parameter for the proposal distribution variation
ir_t = 48;%Impulse response function horizon
mp_shock = zeros(M,1);%Monetary policy shock
mp_shock(M) = 1;
multistarts = 2500;%Number of starts
t_proj = 36;%Variance decomposition
text = 'Prob 2-1-';
% Replace GDP with employment-population ratio
IMF_Errbd(Dat,p,lam,M,T,K,vc,mu,Num,C,ir_t,mp_shock,multistarts,t_proj,text);
% Replace GDP with industrial production
Dat = Dat(:,[9,3:7]);
text = 'Prob 2-2-';
IMF_Errbd(Dat,p,lam,M,T,K,vc,mu,Num,C,ir_t,mp_shock,multistarts,t_proj,text);
% Replace GDP with industrial produciton, and M2 divisia monetary index
% with M2 money stock
Dat = Dat(:,[9,3:5,10,7]);
text = 'Prob 2-3-';
IMF_Errbd(Dat,p,lam,M,T,K,vc,mu,Num,C,ir_t,mp_shock,multistarts,t_proj,text);
\end{lstlisting}

\begin{lstlisting}[language=Matlab ,caption={\textit{prob3.m}}, label={code:prob3}]
clear();
%Load the data set
Dat = xlsread('SZdataExtended.xlsx');
Dat = Dat(:,2:7);
%Set constants
p = 13;%Maximum lag
lam = 0.2;%Hyperparameter
M = 6;%Dimension of vector
T = size(Dat,1)-p;%Time for forcasting
K = M*p+1;
vc = 10^6;%First element in Minnesota prior
mu = 1;
Num = 500000;%Number of MCMC simulation
C = 0.5;%Scaling parameter for the proposal distribution variation
ir_t = 48;%Impulse response function horizon
mp_shock = zeros(M,1);%Monetary policy shock
mp_shock(M) = 1;
multistarts = 2500;%Number of starts
t_proj = 36;%Variance decomposition
text = 'Prob 3-2-';
% Use the entire data set
IMF_Errbd(Dat,p,lam,M,T,K,vc,mu,Num,C,ir_t,mp_shock,multistarts,t_proj,text);
% Excluding ZLB period
Dat = Dat(1:600,2:7);
text = 'Prob 3-1-';
IMF_Errbd(Dat,p,lam,M,T,K,vc,mu,Num,C,ir_t,mp_shock,multistarts,t_proj,text);
\end{lstlisting}

\begin{lstlisting}[language=Matlab ,caption={\textit{IMF\textunderscore Errbd.m} function calculates the mode of posteriors for $A_0$ and $B$. Then, use them calculate impulse response functions with error bands by empirical quantile from MCMC simulation.}, label={code:IMF_Errbd}]
function IMF_Errbd(Dat,p,lam,M,T,K,vc,mu,Num,C,ir_t,mp_shock,multistarts,t_proj,text)

%clear();
%Load the data set
% Dat = xlsread('SZdata.xlsx');
% Dat = Dat(:,2:7);
% %Set constants
% p = 13;
% lam = 0.2;
% M = 6;
% T = size(Dat,1)-p;
% K = M*p+1;
% vc = 10^6;
% mu = 1;
% Num = 100000;%Number of MCMC simulation
% C = 0.6;%Scaling parameter for the proposal distribution variation
% ir_t = 48;
% mp_shock = zeros(M,1);
% mp_shock(6) = 1;
% multistarts = 2500;
% t_proj = 36;
%Create matrix of X and Y combining observations and dummy observations
[X,Y] = createXY(Dat,p,T,M,K,mu);

%Estimate elements of Omega
omega_d = createOmega(Dat,p,T,M,K,vc,lam);

%Estimate the MAP for B and constant for b_hat in prior
b_hat = [zeros(M,1),eye(M,M*p)]';
B_hat = ((X'*X)+diag(omega_d.^(-1)))\(X'*Y + diag(omega_d.^(-1))*b_hat);
csvwrite([text,'B_hat_mine.csv'],B_hat);

%% Optimize posterior for A_0
func = @(a0)-logPostA0(a0,X,Y,omega_d/lam^2,b_hat,B_hat,M,T,lam);
solver = 'fminunc';
options = optimoptions(@fminunc,'Algorithm','quasi-newton','Display', 'off');
%Solve once and obtain hessian matrix for the proposal distribution in
%Metropolis alg
a0_0 = initPoint(X,Y,B_hat,b_hat,omega_d);%a0_0 is a column vector
[a0_init,post_val_init,exitflag,output_init,grad_init,Hessian]=...
    fminunc(func,a0_0,options);

%Use global optimization toolbox to find "global" optimial for the
%posterior of A_0
tic;
filename = 'p1-1.csv';
[a0_opt,manymin] = findGlobalOpt(func,solver,options,multistarts,a0_0,filename);
A0_opt = vec2mat(a0_opt,M);
csvwrite([text,'A0_opt_mine.csv'],A0_opt);
toc;

%% Construct Impulse Response Function
% Monetary policy shock
time_range = [1,ir_t];
tic;
%Sign constraint, so that the one positive monetary shock will result in
%immediate decrease of the federal funds rate (R)
imd_resp = A0_opt\mp_shock;
sign_con = sign((imd_resp(M)));
imf = sign_con*IMF(ir_t,M,p,A0_opt,B_hat,mp_shock,time_range);
toc;
h_errbd = figure();
hold on;
box on;
annotation('textbox', [0 0.9 1 0.1], ...
    'String', [text, 'Error bands of impulse response functions'], ...
    'EdgeColor', 'none', ...
    'HorizontalAlignment', 'center',...
    'FontSize',20);
legendinfo{1}=['Impulse response to Monetary Policy Shock'];
handle(1) = plot_all_TS(imf,'-k',[time_range,-inf,inf]);
%Check the results and they are close

%% Metropolis algorithm and error bands
%acceptance rate
eps_hat = Y - X*B_hat;
S_dhat = eps_hat'*eps_hat+(B_hat-b_hat)'/diag(omega_d)*(B_hat-b_hat);
tic;
[ar,trace_dat] = Metropolis_alg(Num,a0_0,C,Hessian,S_dhat,T,M,text);%Draw A0 from its marginal posterior distribution
toc;
imf_MCMC = zeros(ir_t,M,Num);%Store the IMF for each MCMC draw 
tic;
for i = 1:Num
    A0_temp = vec2mat(trace_dat(i,:)',M);
    inv_A0_temp = inv(A0_temp);
    imd_resp = inv_A0_temp*mp_shock;%Sign constraint as before
    sign_con = sign(imd_resp(M));
    V_temp = kron(inv_A0_temp*inv_A0_temp',inv(X'*X+diag(omega_d.^(-1))));
    B_new = reshape(mvnrnd(reshape(B_hat,K*M,1),V_temp),K,M);%Draw new B based on each draw of A0
    imf_temp = sign_con*IMF(ir_t,M,p,A0_temp,B_new,mp_shock,time_range);%Calculate IMF for this sample of A0 and B
    imf_MCMC(:,:,i) = imf_temp(:,2:end);
end
err_bd = quantile(imf_MCMC,[0.05,0.95,0.16,0.84],3);%Empirical quantile for each IMF
legendinfo{2} = ['%5 error bound'];
handle(2) = plot_all_TS([(1:ir_t)',err_bd(:,:,1)],'b-.',[time_range,-inf,inf]);
legendinfo{3} = ['%95 error bound'];
handle(3) = plot_all_TS([(1:ir_t)',err_bd(:,:,2)],'r-.',[time_range,-inf,inf]);
legendinfo{4} = ['%16 error bound'];
handle(4) = plot_all_TS([(1:ir_t)',err_bd(:,:,3)],'b.',[time_range,-inf,inf]);
legendinfo{5} = ['%84 error bound'];
handle(5) = plot_all_TS([(1:ir_t)',err_bd(:,:,4)],'r.',[time_range,-inf,inf]);
legend(handle,legendinfo,'Location','northeast');
toc;
saveas(gca,[text,'Error_Bands'],'jpg');
saveas(gca,[text,'Error_Bands'],'fig');
close(gcf);

%Trace plot
tic;
h_trace_plot = figure();
hold on;
box on;
annotation('textbox', [0 0.9 1 0.1], ...
    'String', [text,'Trace plots for free variables'], ...
    'EdgeColor', 'none', ...
    'HorizontalAlignment', 'center',...
    'FontSize',20);
handle1 = plot_trace(trace_dat,6,3,'b-');
toc;
saveas(gca,[text,'Trace_plots'],'jpg');
saveas(gca,[text,'Trace_plots'],'fig');
close(gcf);

%% Variance Decomposition
var_imf_mp = imf_var(M,p,A0_opt,B_hat,mp_shock,t_proj);
var_imf_all = imf_var(M,p,A0_opt,B_hat,ones(M,1),t_proj);
display([text,'The portion of variance of GDP due to monetary policy shock is:',10]);
display(var_imf_mp(1,1)/var_imf_all(1,1));
end
\end{lstlisting}

\begin{lstlisting}[language=Matlab ,caption={\textit{createOmega.m} creates $ \Omega$ for Minnesota prior.}, label={code:createOmega}]
function omega_d = createOmega(Dat,p,T,M,K,vc,lam)
%% The first observation is given and the rest of observation up to the
% end of current window are for prediction as stated in the problem
% statement
% Follow Hamilton 1994 to get the MLE for variance
sum1 = sum(Dat(p:p+T-1,:),1);%sum_1^T y_(t-1)
sum2 = sum(Dat(p+1:p+T,:),1);%sum_1^T y_(t)
sum3 = sum(Dat(p:p+T-1,:).^2,1);%sum_1^T y_(t-1)^2
sum4 = sum(Dat(p+1:p+T,:).^2,1);%sum_1^T y_(t)^2
sum5 = sum(Dat(p+1:p+T,:).*Dat(p:p+T-1,:),1);%sum_1^T y_(t)y_(t-1)
sigma2_hat = zeros(1,M);
for m = 1:M
    c_phi_hat = [T,sum1(m);sum1(m),sum3(m)]\[sum2(m);sum5(m)];% c_phi_hat = [c_hat,phi_hat]
    %temp_sum = 0;
    sigma2_hat(m) = 1.0/(T)*(sum4(m)+c_phi_hat(2)^2*sum3(m)+2*c_phi_hat(1)*c_phi_hat(2)*sum1(m)...
                    -2*c_phi_hat(1)*sum2(m)-2*c_phi_hat(2)*sum5(m))+c_phi_hat(1)^2;
end
omega_d = zeros(1,K);
omega_d(1) = [vc/lam^2];
for j = 1:p
    omega_d(1+(j-1)*M+1:1+j*M) = (sigma2_hat*j^2).^(-1);
end
omega_d = lam^2 * omega_d;
end
\end{lstlisting}

\begin{lstlisting}[language=Matlab ,caption={\textit{createXY.m} constructs matrix $X$ and $Y$ with sum-of-coefficient prior with dummy observations.}, label={code:createXY.m}]
function [X,Y] = createXY(Dat,p,T,M,K,mu)
%Assemble the ordinary Y
Y = Dat((p+1:p+T),:);
%Assemble the ordinary X
X = zeros(T,K);% a T*K matrix
for j = 1:T
    X(j,:) = [1,reshape(Dat((j+p-1:-1:j),:)',1,K-1)];% have to transpose because the reshape function operate in column
end

%Add sum-of-coefficient
Y_bar = diag(mu*mean(Dat(1:p,:),1));
Y = [Y;Y_bar];
Xplus = [zeros(M,1),repmat(diag(mu*mean(Dat(1:p,:),1)),[1 p])];    
X= [X;Xplus]; % stack x at the bottom
end
\end{lstlisting}

\begin{lstlisting}[language=Matlab ,caption={\textit{findGlobalOpt.m} uses multi-starts in global optimization toolbox to find somewhat ``global" optimal value for matrix $A_0$ by maximizing the marginal posterior \ref{eqn:post_A0}}, label={code:findGlobalOpt}]
function [a0_opt, manymin] = findGlobalOpt(objFunc,solver,options,multistarts,a0_0,filename)
%% Use Global Optimization Toolbox in Matlab to find a "global" optimal of
%the objective function.
%Input: 
    %objFunc: The objFunc has to have defined the variables and parameters
%========================================
    %Input:
        %lam: lambda;
        %A0: The A_0 matrix (M-by-M);
        %X: X matrix ((T+M)-by-K);
        %Y: Y matrix ((T+M)-by-M);
        %d_phi: Diagonal entries for square root of Phi matrix (K entries)
        %b_hat: Matrix form for the mean of prior for beta (which is a vector form of B);
        %M: Dimension of vector variables;
        %T: Time periods for forcasting;
        %p: The maximum lag;
    %Output:
        %logfval: The log posterior function value;
%========================================
    %solver: The specified solver. Here use 'fminunc';
    %options: Options for the solver (optimoptions(@fminunc,'Algorithm','quasi-newton','Display', 'off')).
    %          Notice that here we are necessarily requiring gradient of objFunc
    %multistarts: Number of multistart;
    %a0_0: The initial guess for a_0;
    %filename: The file name for writing the resutls
%Output:
    %a0_opt: The optimal a0 with the optimal function value.
%% ==========================================================================
%Create an optimization problem
prob = createOptimProblem(solver,'objective',objFunc,'x0',a0_0,'options',options);
%Number of multistart
MS = MultiStart;
%Run multistarts times of this problem
[x,f,~,~,manymin] = run(MS,prob,multistarts);
%Write to a file and get the optimal value for free variables
csvwrite(filename,x);
a0_opt = x;
end
\end{lstlisting}

\begin{lstlisting}[language=Matlab ,caption={\textit{IMF.m} constructs impulse response functions using companion form.}, label={code:IMF}]
function imf = IMF(T,M,p,A0,B_hat,mp_shock,time_range)
%% Use companion form to get the impulse response function interms of time.
%Input:
%   T: The time horizon to investigate the impulse response;
%   M: Dimension of vector variable;
%   p: The maximum time lag;
%   A0: The MAP of A0 matrix (M-by-M);
%   B_hat: The MAP of B matrix (K-by-M);
%   mp_shock: To specify which shock is in interest (M-by-1);
%Output:
%   imf: Impulse response function in terms of time periods (T-by-M);

% Phi is a K-by-K matrix
Phi = sparse([B_hat(2:end,:)';eye((p-1)*M,p*M)]);
% Big shock column: K-by-1
MP_shock = sparse([mp_shock;zeros((p-1)*M,1)]);
% Big matrix pre-multiplied on shocks
G = sparse(zeros(p*M,p*M));
G(1:M,1:M) = A0\eye(M);

imf = zeros(T,M);
temp = G*MP_shock;
for i = 1:T
   temp = Phi*temp;
   imf(i,:) = temp(1:M)';
end
imf = [(time_range(1):(time_range(2)-time_range(1))/(T-1):time_range(2))',imf];
end
\end{lstlisting}

\begin{lstlisting}[language=Matlab ,caption={\textit{initPoint.m} returns initial guess for global optimization executed by \textit{findGlobalOpt.m}.}, label={code:initPoint}]
function a0_0 = initPoint(X,Y,B_hat,b_hat,omega_d)
%Residual;
eps_hat = Y - X*B_hat;
%Another part;
temp1 = (diag(omega_d.^0.5)\(B_hat-b_hat));
%S_hat
S_hat = eps_hat'*eps_hat+temp1'*temp1;
%Cholesky decomposition on S_hat'*S_hat to get the initial A0 and then
%return the vector version of it.
A0_0 = chol(S_hat'*S_hat,'upper');
a0_0 = mat2vec(A0_0);
end
\end{lstlisting}

\begin{lstlisting}[language=Matlab ,caption={\textit{logPostA0.m} calculates log-posterior of $A_0$.}, label={code:logPostA0}]
function logfval = logPostA0(a0,X,Y,d_phi,b_hat,B_hat,M,T,lam)
%% Calculate the log posterior for A_0 and return the B_hat
%Input:
%lam: lambda;
%A0: The A_0 matrix (M-by-M);
%X: X matrix ((T+M)-by-K);
%Y: Y matrix ((T+M)-by-M);
%d_phi: Diagonal entries for square root of Phi matrix (K entries)
%b_hat: Matrix form for the mean of prior for beta (which is a vector form of B);
%B_hat: Matrix of coefficient (K-by-M);
%M: Dimension of vector variables;
%T: Time periods for forcasting;
%p: The maximum lag;
%Output:
%logfval: The log posterior function value;

%% ==========================================================================
%MAP for B;
% B_hat = (lam^2*(X'*X)+diag(d_phi.^(-1)))\(lam^2*X'*Y + diag(d_phi.^(-1))*b_hat);
%Residual;
eps_hat = Y - X*B_hat;
%Assemble the A0 from a0
A0 = vec2mat(a0,M);
%A simple way to calculate the arguments inside exponant;
% temp1 = eps_hat*A0';
% temp2 = lam^(-1)*(diag(d_phi.^0.5)\(B_hat-b_hat))*A0';
%The trace of a matrix times its transpose is just the Frobenius norm of
%that matrix;
%logfval = (T+M)*log(det(A0))-0.5*(sum(sum(temp1.^2))+sum(sum(temp2.^2))); %equivalent form
logfval = (T+M)*log(det(A0))-0.5*trace((eps_hat'*eps_hat+lam^(-2)*(B_hat-b_hat)'/diag(d_phi)*(B_hat-b_hat))*(A0'*A0));
end
\end{lstlisting}

\begin{lstlisting}[language=Matlab ,caption={\textit{mat2vec.m} transfers a vector $a_0$ of free variables in matrix $A_0$ into a matrix $A_0$.}, label={code:mat2vec}]
function a = mat2vec(A)
[n,m] = size(A);
A = reshape(A,n*m,1);
a = A([1,2,3,4,5,8,9,10,11,15,16,22,28,29,30,34,35,36]);
end
\end{lstlisting}

\begin{lstlisting}[language=Matlab ,caption={\textit{Metropolis\textunderscore alg.m} implements MCMC using Metropolis algorithm.}, label={code:Metropolis_alg}]
function [ar,trace_dat] = Metropolis_alg(Num,a0_0,C,Hessian,S_dhat,T,M,text)

if (issymmetric(Hessian)~=1)
    Hessian = (Hessian+Hessian')/2;
end
delta = min(eig(Hessian));
if (delta < 0)
    Hessian = Hessian -2*delta*eye(size(Hessian,1));
end

V = C^2*inv(Hessian);
trace_dat = zeros(Num,length(a0_0));
accept = 0;
for i = 1:Num
    A0_0 = vec2mat(a0_0,M);
    a0_new = mvnrnd(a0_0,V);
    A0_new = vec2mat(a0_new,M);
    lgp_0 = (T+M)*log(det(A0_0))-0.5*trace(S_dhat*(A0_0'*A0_0));
    lgp_new = (T+M)*log(det(A0_new))-0.5*trace(S_dhat*(A0_new'*A0_new));
    rate = exp(lgp_new-lgp_0);
    if rate >= 1
        a0_0 = a0_new;
        accept = accept + 1;
    else
        u = rand;
        if u <= rate
            a0_0 = a0_new;
            accept = accept + 1;
        end
    end
    trace_dat(i,:) = a0_0';
end
ar = accept / Num;
csvwrite([text,'trace_plot.csv'],trace_dat);
end
\end{lstlisting}

\begin{lstlisting}[language=Matlab ,caption={\textit{plot\textunderscore all\textunderscore TS} plots impulse response functions (w.r.t $t$) into subplots.}, label={code:plot_all_TS}]
function h=plot_all_TS(TSdat,patten,axlmt)
    subplot(3,2,4);
    hold on;
    box on;
    grid on;
    plot(abs(TSdat(:,1)),TSdat(:,2),patten);
    axis(axlmt)
    % legend('Observed Data','DC Prediction')
    %legend(h,text)
    title('Y')

    subplot(3,2,5);
    hold on;
    box on;
    grid on;
    plot(abs(TSdat(:,1)),TSdat(:,3),patten);
    axis(axlmt)
    % legend('Observed Data','DC Prediction')
    title('P')
    
    subplot(3,2,6);
    hold on;
    box on;
    grid on;
    plot(abs(TSdat(:,1)),TSdat(:,4),patten);%,xt,rt2.predgdp,'-.m',xt,rt3.predgdp,'--c',xt,rt4.predgdp,':r')
    axis(axlmt)
    % legend('Observed Data','DC Prediction')
    title('U')
    
    subplot(3,2,1);
    hold on;
    box on;
    grid on;
    plot(abs(TSdat(:,1)),TSdat(:,5),patten);%,xt,rt2.predgdp,'-.m',xt,rt3.predgdp,'--c',xt,rt4.predgdp,':r')
    axis(axlmt)
    % legend('Observed Data','DC Prediction')
    title('Pcom')
    
    subplot(3,2,2);
    hold on;
    box on;
    grid on;
    plot(abs(TSdat(:,1)),TSdat(:,6),patten);%,xt,rt2.predgdp,'-.m',xt,rt3.predgdp,'--c',xt,rt4.predgdp,':r')
    axis(axlmt)
    % legend('Observed Data','DC Prediction')
    title('M2')
    
    subplot(3,2,3);
    hold on;
    box on;
    grid on;
    h=plot(abs(TSdat(:,1)),TSdat(:,7),patten);%,xt,rt2.predgdp,'-.m',xt,rt3.predgdp,'--c',xt,rt4.predgdp,':r')
    axis(axlmt)
    % legend('Observed Data','DC Prediction')
    title('R')
end
\end{lstlisting}

\begin{lstlisting}[language=Matlab ,caption={\textit{plot\textunderscore trace.m} plots trace plots of free variables in vector $a_0$.}, label={code:plot_trace}]
function h=plot_trace(trace_dat,n,m,patten)

[Num,~] = size(trace_dat);

for i = 1:(n*m)

subplot(n,m,i);
hold on;
box on;
h=plot(1:Num,trace_dat(:,i),patten);
% legend('Observed Data','DC Prediction')
%legend(h,text)
title(['Free variable ',num2str(i)]);
end
end
\end{lstlisting}

\begin{lstlisting}[language=Matlab ,caption={\textit{vec2mat.m} transfers the free variables in $A_0$ into a vector $a_0$.}, label={code:vec2mat}]
function A = vec2mat(a,M)
% A = zeros(M^2,1);
% A([1,7,13,19,25,8,14,20,26,15,21,22,23,29,35,24,30,36]) = a;
% A = reshape(A,M,M);
% A = A';

A = zeros(M^2,1);
A([1,2,3,4,5,8,9,10,11,15,16,22,28,29,30,34,35,36]) = a;
A = reshape(A,M,M);
end
\end{lstlisting}

\begin{lstlisting}[language=Matlab ,caption={\textit{imf\textunderscore var.m} calculates the IMF variance based on shock given.}, label={code:imf_var}]
function IMF_Errbd(Dat,p,lam,M,T,K,vc,mu,Num,C,ir_t,mp_shock,multistarts,t_proj,text)

%clear();
%Load the data set
% Dat = xlsread('SZdata.xlsx');
% Dat = Dat(:,2:7);
% %Set constants
% p = 13;
% lam = 0.2;
% M = 6;
% T = size(Dat,1)-p;
% K = M*p+1;
% vc = 10^6;
% mu = 1;
% Num = 100000;%Number of MCMC simulation
% C = 0.6;%Scaling parameter for the proposal distribution variation
% ir_t = 48;
% mp_shock = zeros(M,1);
% mp_shock(6) = 1;
% multistarts = 2500;
% t_proj = 36;
%Create matrix of X and Y combining observations and dummy observations
[X,Y] = createXY(Dat,p,T,M,K,mu);

%Estimate elements of Omega
omega_d = createOmega(Dat,p,T,M,K,vc,lam);

%Estimate the MAP for B and constant for b_hat in prior
b_hat = [zeros(M,1),eye(M,M*p)]';
B_hat = ((X'*X)+diag(omega_d.^(-1)))\(X'*Y + diag(omega_d.^(-1))*b_hat);
csvwrite([text,'B_hat_mine.csv'],B_hat);

%% Optimize posterior for A_0
func = @(a0)-logPostA0(a0,X,Y,omega_d/lam^2,b_hat,B_hat,M,T,lam);
solver = 'fminunc';
options = optimoptions(@fminunc,'Algorithm','quasi-newton','Display', 'off');
%Solve once and obtain hessian matrix for the proposal distribution in
%Metropolis alg
a0_0 = initPoint(X,Y,B_hat,b_hat,omega_d);%a0_0 is a column vector
[a0_init,post_val_init,exitflag,output_init,grad_init,Hessian]=...
    fminunc(func,a0_0,options);

%Use global optimization toolbox to find "global" optimial for the
%posterior of A_0
tic;
filename = 'p1-1.csv';
[a0_opt,manymin] = findGlobalOpt(func,solver,options,multistarts,a0_0,filename);
A0_opt = vec2mat(a0_opt,M);
csvwrite([text,'A0_opt_mine.csv'],A0_opt);
toc;

%% Construct Impulse Response Function
% Monetary policy shock
time_range = [1,ir_t];
tic;
%Sign constraint, so that the one positive monetary shock will result in
%immediate decrease of the federal funds rate (R)
imd_resp = A0_opt\mp_shock;
sign_con = sign((imd_resp(M)));
imf = sign_con*IMF(ir_t,M,p,A0_opt,B_hat,mp_shock,time_range);
toc;
h_errbd = figure(1);
hold on;
box on;
annotation('textbox', [0 0.9 1 0.1], ...
    'String', [text, 'Error bands of impulse response functions'], ...
    'EdgeColor', 'none', ...
    'HorizontalAlignment', 'center',...
    'FontSize',20);
legendinfo{1}=['Impulse response to Monetary Policy Shock'];
handle(1) = plot_all_TS(imf,'-k',[time_range,-inf,inf]);
%Check the results and they are close

%% Metropolis algorithm and error bands
%acceptance rate
eps_hat = Y - X*B_hat;
S_dhat = eps_hat'*eps_hat+(B_hat-b_hat)'/diag(omega_d)*(B_hat-b_hat);
tic;
[ar,trace_dat] = Metropolis_alg(Num,a0_0,C,Hessian,S_dhat,T,M);%Draw A0 from its marginal posterior distribution
toc;
imf_MCMC = zeros(ir_t,M,Num);%Store the IMF for each MCMC draw 
tic;
for i = 1:Num
    A0_temp = vec2mat(trace_dat(i,:)',M);
    inv_A0_temp = inv(A0_temp);
    imd_resp = inv_A0_temp*mp_shock;%Sign constraint as before
    sign_con = sign(imd_resp(M));
    V_temp = kron(inv_A0_temp*inv_A0_temp',inv(X'*X+diag(omega_d.^(-1))));
    B_new = reshape(mvnrnd(reshape(B_hat,K*M,1),V_temp),K,M);%Draw new B based on each draw of A0
    imf_temp = sign_con*IMF(ir_t,M,p,A0_temp,B_new,mp_shock,time_range);%Calculate IMF for this sample of A0 and B
    imf_MCMC(:,:,i) = imf_temp(:,2:end);
end
err_bd = quantile(imf_MCMC,[0.05,0.95,0.16,0.84],3);%Empirical quantile for each IMF
legendinfo{2} = ['%5 error bound'];
handle(2) = plot_all_TS([(1:ir_t)',err_bd(:,:,1)],'b-.',[time_range,-inf,inf]);
legendinfo{3} = ['%95 error bound'];
handle(3) = plot_all_TS([(1:ir_t)',err_bd(:,:,2)],'r-.',[time_range,-inf,inf]);
legendinfo{4} = ['%16 error bound'];
handle(4) = plot_all_TS([(1:ir_t)',err_bd(:,:,3)],'b.',[time_range,-inf,inf]);
legendinfo{5} = ['%84 error bound'];
handle(5) = plot_all_TS([(1:ir_t)',err_bd(:,:,4)],'r.',[time_range,-inf,inf]);
legend(handle,legendinfo,'Location','northeast');
toc;
saveas(h_errbd,[text,'Error_Bands'],'jpg');
saveas(h_errbd,[text,'Error_Bands'],'fig');

%Trace plot
tic;
h_trace_plot = figure(2);
hold on;
box on;
annotation('textbox', [0 0.9 1 0.1], ...
    'String', [text,'Trace plots for free variables'], ...
    'EdgeColor', 'none', ...
    'HorizontalAlignment', 'center',...
    'FontSize',20);
handle1 = plot_trace(trace_dat,6,3,'b-');
toc;
saveas(h_trace_plot,[text,'Trace_plots'],'jpg');
saveas(h_trace_plot,[text,'Trace_plots'],'fig');

%% Variance Decomposition
var_imf_mp = imf_var(M,p,A0_opt,B_hat,mp_shock,t_proj);
var_imf_all = imf_var(M,p,A0_opt,B_hat,ones(M,1),t_proj);
display([text,'The portion of variance of GDP due to monetary policy shock is:',10]);
display(var_imf_mp(1,1)/var_imf_all(1,1));
end
\end{lstlisting}

\end{appendices}
\end{document}  